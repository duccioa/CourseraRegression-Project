\documentclass[]{article}
\usepackage{lmodern}
\usepackage{amssymb,amsmath}
\usepackage{ifxetex,ifluatex}
\usepackage{fixltx2e} % provides \textsubscript
\ifnum 0\ifxetex 1\fi\ifluatex 1\fi=0 % if pdftex
  \usepackage[T1]{fontenc}
  \usepackage[utf8]{inputenc}
\else % if luatex or xelatex
  \ifxetex
    \usepackage{mathspec}
    \usepackage{xltxtra,xunicode}
  \else
    \usepackage{fontspec}
  \fi
  \defaultfontfeatures{Mapping=tex-text,Scale=MatchLowercase}
  \newcommand{\euro}{€}
\fi
% use upquote if available, for straight quotes in verbatim environments
\IfFileExists{upquote.sty}{\usepackage{upquote}}{}
% use microtype if available
\IfFileExists{microtype.sty}{%
\usepackage{microtype}
\UseMicrotypeSet[protrusion]{basicmath} % disable protrusion for tt fonts
}{}
\usepackage[margin=1in]{geometry}
\ifxetex
  \usepackage[setpagesize=false, % page size defined by xetex
              unicode=false, % unicode breaks when used with xetex
              xetex]{hyperref}
\else
  \usepackage[unicode=true]{hyperref}
\fi
\hypersetup{breaklinks=true,
            bookmarks=true,
            pdfauthor={},
            pdftitle={},
            colorlinks=true,
            citecolor=blue,
            urlcolor=blue,
            linkcolor=magenta,
            pdfborder={0 0 0}}
\urlstyle{same}  % don't use monospace font for urls
\setlength{\parindent}{0pt}
\setlength{\parskip}{6pt plus 2pt minus 1pt}
\setlength{\emergencystretch}{3em}  % prevent overfull lines
\setcounter{secnumdepth}{0}

%%% Use protect on footnotes to avoid problems with footnotes in titles
\let\rmarkdownfootnote\footnote%
\def\footnote{\protect\rmarkdownfootnote}

%%% Change title format to be more compact
\usepackage{titling}

% Create subtitle command for use in maketitle
\newcommand{\subtitle}[1]{
  \posttitle{
    \begin{center}\large#1\end{center}
    }
}

\setlength{\droptitle}{-2em}
  \title{}
  \pretitle{\vspace{\droptitle}}
  \posttitle{}
  \author{}
  \preauthor{}\postauthor{}
  \date{}
  \predate{}\postdate{}



\begin{document}

\maketitle


\textbf{\emph{EXECUTIVE SUMMARY}}

This report aims at investigating the following questions:

\begin{enumerate}
\def\labelenumi{\arabic{enumi}.}
\itemsep1pt\parskip0pt\parsep0pt
\item
  Is an automatic or manual transmission better for MPG (miles per
  gallons)\\
\item
  Quantify the MPG difference between automatic and manual transmissions
\end{enumerate}

The analysis is carried out on the dataset mtcars. Due to the small
number of models and the age of the dataset, the results cannot be
generalised to the current population of cars, but can give an insight
within the set of cars considered.

The sole distinction between manual and automatic cars is a poor
predictor of the fuel consumption and other factors more influential
have to be taken into account in a more precise prediction. The other
reason we need to consider other factors is that in the database, manual
cars tend to be imported models generally lighter and smaller, hence
with lower fuel consumption.

To better answer the question we therefore propose a model to predict
fuel consumption which includes, on top of the a/m distinction, the
weight of the car and the number of cylinders. In this context, there is
not a statistically significant difference in fuel consumption between
manual cars and automatic ones.

\textbf{\emph{REPORT}}

The database contains \textbf{\emph{32}} car models of which
\textbf{\emph{19}} have automatic transmission and \textbf{\emph{13}}
have manual. The variable ``am'' is \textbf{\emph{0}} for automatic
transmission and \textbf{\emph{1}} for manual.

A simple t-test confirm that the two transmissions are not from the same
population as the interval does not contain 0 and the p-value
\textbf{\emph{0}} is sufficiently low (see appendix 2). For the sake of
this projects we assume ``mpg'' to have a normal distribution, although
it looks a little skewed.

We consider a linear model with ``am'' as a predictor and ``mpg'' as a
response. The relatevly high F-statistic and the very low p-value
confirm that there might be a correlation, but if we look at how well
the model fit, R-squared \textbf{\emph{0.36}} is very low (1 being best
fit, 0 being no fit at all). This, together with a Residual Standard
Error (RSE) of ***4.822, suggests that the model is not very accurate
and that there might be other model to better explain the correlation.

We look therefore to other variables that might be confounder. Based on
a brief literature review, the variables that can be relevant to our
analysis are weight, the number of cylinders and the rear axle ratio
(drat in dataset). We can call primary this variables as this are the
engines' and cars' specifications rather than performances. Other
variables as ``qsec'' (which measures the acceleration) and ``hp''
(which measure the power of the engine) can be good indicators but they
are performance variables and are consequences of the primary specs.
Weight and the number of cylinders can tell a lot about a car because
they are related to the size and power of the engine and therefore the
size and the type of car. Displacement (the amount of fuel burnt per
stroke of the engine) gives similar information about the engine but it
is correlated with cylinders (see Appendix \textbf{\emph{XX}}).

A look at the ``wt'' plot of Appendix 1 suggests that manuals cars in
the database are generally lighter. In Appendix 3, we plot the two
different regression lines for manual cars (in red) and automatic (in
black) with ``wt'' as predictor and ``mpg'' as an outcome. From the
graph it seems that weight affects manual trasmission more than
automatic transmission. Based on literature review we disregard this
hypothesis and we suggest that this is due to ``wt'' being a quadratic
predictor of ``mpg'' (that is, the first tons impact fuel consumption
more than farther ones) and manual/lighter cars of this dataset, being
in the left part of the graph, have a quicker decline in performance
than automatic ones when weight increases. Indeed a linear model with a
quadratic predictor better explain the relationship between ``wt'' and
``mpg'', as the R-squared goes from \textbf{\emph{0.75}} to
\textbf{\emph{0.82}} with a reduced RSE from \textbf{\emph{3.05}} to
\textbf{\emph{2.65}} and improved p-value (see appendix 4)

Once we factor in the weight ``wt'', the model explains \textbf{\emph{r
paste(round(summary(fit\_cyl\_wt)\$r.squared}100, 1), ``\%'')}* of
``mpg'' variation with a RSE of and very low p-values. In the final part
we add the variable ``am''. From the graph in the appendix with ``am''
as a predictor, it seems that there is an interaction between ``am'' and
``wt'', as automatic cars in the dataset tend to be heavier than the
manuals one (a t.test over the two populations confirm that, see
APPENDIX).

The third model includes cyinders, weight, manual/automatic transmission
and the interaction between the latter and the weight

\end{document}
